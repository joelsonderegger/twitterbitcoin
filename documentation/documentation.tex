\documentclass[a4paper,12pt]{article}
\usepackage{graphicx}
\graphicspath{ {images/} }

%% Language and font encodings
\usepackage[utf8]{inputenc}
\usepackage[american]{babel}

%% Sets page size and margins
\usepackage[top=2cm,bottom=2cm,left=2cm,right=2cm,marginparwidth=1.75cm]{geometry}
\setlength{\parskip}{1em}

%% package for formatting and highlighting source code
\usepackage{listings}
\usepackage{color}


%% Useful packages
\usepackage{amsmath}
\usepackage{graphicx}
\usepackage{float}
\usepackage[colorlinks=true, allcolors=black]{hyperref}
\usepackage{graphicx}
\usepackage{setspace}

% List of abbreviations
\usepackage{acronym}



% Specify bibliography package
\usepackage{csquotes}% Recommended
\usepackage[style=apa]{biblatex}
%\DeclareLanguageMapping{american}{american-apa}
\addbibresource{references.bib}

% Section Title Margins
\usepackage{titlesec}
\titlespacing*{\section}
{0pt}{3.0ex plus 1ex minus .2ex}{0.7ex}
\titlespacing*{\subsection}
{0pt}{1.4ex}{0pt}

\definecolor{dkgreen}{rgb}{0,0.6,0}
\definecolor{gray}{rgb}{0.5,0.5,0.5}
\definecolor{mauve}{rgb}{0.58,0,0.82}

\lstset{frame=tb,
  language=Python,
  aboveskip=3mm,
  belowskip=3mm,
  showstringspaces=false,
  columns=flexible,
  basicstyle={\small\ttfamily},
  numbers=none,
  numberstyle=\tiny\color{gray},
  keywordstyle=\color{blue},
  commentstyle=\color{dkgreen},
  stringstyle=\color{mauve},
  breaklines=true,
  breakatwhitespace=true,
  tabsize=3
}
    
    
\title{Review of Assigned Readings}
\author{Your Name Goes Here}
        
    \begin{document}
    \begin{titlepage}
      	\begin{figure}
      	\centering
        \includegraphics[scale=0.35]{logohsg}
        \end{figure}
        \centering
        School of Management, Economics, Law, Social Schiences and International Affairs \par
        \vspace{2.5cm}
        {\huge\bfseries The effect of Twitter activity on
Bitcoin price\par}      
        {\Huge\itshape Documentation\par}
        \vspace{1.5cm}     
        {Software Engineering for Economists \\ (7,610,1.00) \par}
        \vspace{1.3cm}
        {Dimitrios Koumnakes - 10-613-370 \\ Severin Kranz - 13-606-355 \\ Joël Sonderegger - 11-495-488 \\
       	Alen Stepic - 11-475-258 \\Chi Xu - XX-XXX-XXX \par}
        \vspace{1cm}
        Fall Term 2017
        \vfill
        Supervisor
        \linebreak
        Prof. Dr. Philipp \textsc{Zahn}
        \linebreak
        Department of Economics
          
    % Bottom of the page
     {\centering \today\par}
    
    \end{titlepage}
    
    \begin{abstract}
    \pagenumbering{Roman}
    \begin{spacing}{1.5}
    
  (Insert text)

    \end{spacing}
    
    \end{abstract}

    \clearpage
    \tableofcontents
    
    \clearpage
    
    \listoffigures 	
    \section*{List of Abbreviation} 
	\begin{acronym}[ASECRETTT] 
	\acro{API}{Application Programming Interface}
	\acro{CKEY}{Consumer Key}
	\acro{CSECRET}{Consumer Secret}
	\acro{ATOKEN}{Access Token}
	\acro{ASECRET}{Access Token Secret}
	\end{acronym}
    \clearpage
    



\begin{spacing}{1.2}
\cleardoublepage\pagenumbering{arabic}
\section{Introduction}
<<<<<<< HEAD
In the academic environment accountability and reproducibility is important. However, the publishing process of papers and journals seem to be outdated. New ways of data collection and data processing exist by using computational economics. The usage of algorithms can increase effectiveness and efficiency. Hence, much lager data sets can be proceeded. However this creates also new problems regarding to traceability and reproducibility. Often cited academic paper exist, where the initial computation is not reproducible. Furthermore, some academic paper even contain computational errors. Replicating data or existing results do not provide any new knowledge at all. Nevertheless, the ability to reproduce increases trustworthiness and indicate the quality of the conducted work. This explains why reproduction is of great relevance.
=======
In academics accountability and reproducibility is important. However, the publishing process of papers and journals seem to be outdated, as many new ways of data set and processing exist. Examples of academic papers, which are often cited exist where the initial computation are not reproducible and contain errors. Even though just replicating data or existing results do not provide any new knowledge at all, it gives a good indication about the quality of the work done and hence increases trustworthiness. This explains why reproduction is of great relevance and point out the existing problems.
>>>>>>> 7563583098529c04751c3a6630e7b47768bccd0c

\subsection{Goal of the paper}
The goal of this documentation is the provision of a description. This description should enable the reader to reproduce the results discussed in the separate paper. Thus, it contains an explanation how the input data have been gathered, stored, aggregated and analysed. In other words, the input data, the model core, the model parameters and the applied math program are explained.

\subsection{Methodology}
This documentation consists out of four chapters. The first chapter contains a short introduction and provides the reader with an overview about the topic. Furthermore it points out the relevance of documentation. The second chapter discusses the input data. This includes the process of gathering and storing twitter tweets as well as the gathering of the bitcoin price data. The third chapter discusses how the data is aggregated by pointing out the core model and its parameters. Finally the fifth chapter discusses how the analysis has been conducted.

\subsection{Scope}
The scope of the documentation is the provision of an overview about the different steps which have been conducted to obtain the results in the paper. It does not contain any discussions about the results of the separate paper. It is not a deep description of the code as the code itself as the code is documented separately. Nevertheless, important lines of code are discussed. 
\clearpage 


\section{Data Collection}
Here, we provide a detailed description of how the data for the sequential analysis is gathered and stored. This includes two  subsections the (1) tweets data and the (2) bitcoin price data.
\subsection{Tweets Data}
Whit the python script real-time twitter data are streamed and stored. This happens with help of a raspberry pi.
\subsubsection{Python Script}
Twitter offers different Application Programming Interfaces (API) for collecting data. However, the time frame for gathering data on a free base is limited to 7 days. On the other hand, python offers different twitter libraries. Such as the open-source package tweepy. This package has been used for streaming the twitter data as it simplify the script.

\paragraph{Installing Tweepy}\mbox{}\\{}
Tweepy can be installed very simple by running following commands in the command prompt. 
\begin{lstlisting}[language=bash]
pip install tweepy
\end{lstlisting}

If the previous downloaded python installation package does not contain the tweepy library, the tweepy library has to be downloaded. The package can be downloaded for free from the following link:
\begin{lstlisting}[language=bash]
https://pypi.python.org/pypi/tweepy
\end{lstlisting}

\paragraph{Twitter Authentication}\mbox{}\\{}
To access the twitter data, twitter requests an identification of the user. The identification is assured by different keys and access tokens. Those are the (1) consumer key (ckey), (2) consumer secret (csecret), (3) access token (atoken) and (4) access token secret (asecret).

To get the mentioned key and tokens a twitter account is needed. Once, an twitter account exist an application has to be created. This has to be conducted by login with the twitter account credentials under the following link: 
\begin{lstlisting}[language=bash]
https://apps.twitter.com/
\end{lstlisting}

After the creation of an application the keys and tokens can be extracted. Figure 1 illustrates how to retrieve the keys and tokens.
\begin{figure}[h]
\centering
\includegraphics[scale=0.6]{twitteraccess}
\caption{Twitter Keys and Access Tokens}
\end{figure}

\paragraph{Twitter Streaming API}\mbox{}\\{}
By running the python script \verb|collectTwitterData.py| real-time twitter data is pushed in a JSON format. Tweets are pushed just in case the tweet contains the defined key word \verb|bitcoin|.
\begin{lstlisting}[language=bash]
$ python collectTwitterData.py
\end{lstlisting}

From the JSON format the following parameters are decoded
\begin{itemize}
    \item created at: Timestamp of the created tweet
    \item text: text of the tweet
\end{itemize}
The time timestamps is UTC time.

\subsubsection{Hardware Setup}
(Severin's Part)

\subsection{Bitcoin Price Data}
We wrote a Python script which collects Bitcoin price data as there was no preexisting data set that satisfied our needs. The Bitcoin price is best expressed by the Bitcoin Price Index. The Bitcoin price index (BPI) is an index of the exchange rate between the Bitcoin (BTC) and the US dollar (USD) (\cite{kristoufek2015main}). The objective of the script was to gather hourly Bitcoin Price Index data for at least the time period in which we gather the tweets data. We found the an API by bitcoinaverage.com which sufficed our needs. An API description follows later.
\subsubsection{Execution}
\cite{Twitter}
By executing the python script \verb|collectCryptocurrencyData.py| hourly data for the Bitcoin Price Index is retrieved.
\begin{lstlisting}[language=bash]
    $ python collectCryptocurrencyData.py
\end{lstlisting}

\subsubsection{Output}
After successfully running the python script \verb|CollectCryptocurrencyData.py| the file \verb|bpi.csv| is generated in the folder \verb|/data|. It is important to note that every execution of the script overwrites any existing \verb|bpi.csv| file.

The file \verb|bpi.csv| contains historical Bitcoin Price Index data for one month on an hourly basis. Each data point consists of the following parameters:
\begin{itemize}
    \item time: Timestamp on an hourly basis in UTC time
    \item average: Average price (in USD)
    \item high: Highest price (in USD)
    \item low: Lowest price (in USD)
    \item open: Opening price (in USD)
\end{itemize}

\subsubsection{API: Bitcoinaverage.com}
Bitcoinaverage.com offers a free API that provides real-time and historical price data for a range of crypto-currencies including Bitcoin. The following requests delivers data for an per hour monthly sliding window.
\paragraph{Request}\mbox{}\\
The request to get the data for an per hour monthly sliding window looks as follows. This request require authentication that requires registration and the generation of an API key. The registration and generation of an API key is freely available on bitcoinaverage.com. The collectCryptocurrencyData.py already contains the necessary keys. This means that you need no register or generate keys to execute the script collectCryptocurrencyData.py.  
\begin{lstlisting}[language=bash]
https://apiv2.bitcoinaverage.com/indices/global/history/BTCUSD?period=monthly&?format=json
\end{lstlisting}
\paragraph{Response}
An excerpt of an example response looks like the following:
\begin{lstlisting}
[
    {
        "high": 8271.04, 
        "average": 8247.83, 
        "open": 8242.39,
        "low": 8217.72, 
        "time": "2017-11-22 15:00:00"
    }, 
    {
        "high": 8246.82,
        "average": 8203.19,
        "open": 8203.81,
        "low": 8157.25,
        "time": "2017-11-22 14:00:00"
    }, 
    {
        "high": 8267.27, 
        "average": 8238.62, 
        "open": 8248.77, 
        "low": 8198.54, 
        "time": "2017-11-22 13:00:00"
    }
]
\end{lstlisting}


\section{Data Aggregation}
After the two data sets (tweets and Bitcoin Price Index) are gathered


\section{Data Analysis}
(Dimitri's Part)


\end{spacing}
\clearpage

%************************** Bibliography **************************
\printbibliography
\clearpage

%************************** DOCUMENT-END **************************
\end{document}