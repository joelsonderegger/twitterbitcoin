\documentclass[a4paper,american,12pt]{article}
\usepackage{graphicx}
\graphicspath{{images/}}

%************************** Language and font encodings **************************
\usepackage[utf8]{inputenc}

%************************** Sets page size and margins **************************
\usepackage[top=2cm,bottom=2cm,left=2cm,right=2cm,marginparwidth=1.75cm]{geometry}
\setlength{\parskip}{1em}

%************************** Useful packages **************************
\usepackage{amsmath}
\usepackage{graphicx}
\usepackage{subcaption}
\usepackage{float}
\usepackage[colorlinks=true, allcolors=black]{hyperref}
\usepackage{csquotes}
\usepackage{graphicx}
\usepackage{setspace}
\usepackage[noconfigs,british]{babel} 

%************************** List of abbreviations **************************
\usepackage{acronym}

%************************** Specify bibliography package **************************
\usepackage{csquotes}% Recommended
\usepackage[style=apa]{biblatex}
\DeclareLanguageMapping{american}{american-apa}
\addbibresource{references.bib}

%************************** Section Title Margins **************************
\usepackage{titlesec}
\titlespacing*{\section}
{0pt}{3.0ex plus 1ex minus .2ex}{0.7ex}
\titlespacing*{\subsection}
{0pt}{1.4ex}{0pt}
\title{Review of Assigned Readings}
\author{Your Name Goes Here}
    
    
%************************** DOCUMENT_STARTS_HERE **************************
\begin{document}

%************************** Title Page **************************   
\begin{titlepage}
	\begin{figure}
	\centering
	\includegraphics[scale=0.32]{logohsg}
	\end{figure}
\centering
{\scshape\large School of Management, Economics, Law, Social Sciences and International Affairs \par}
\vspace{2.0cm}
{\huge\bfseries The effect of Twitter activity on Bitcoin price fluctuation  \par}
\vspace{2.0cm}
{\scshape\Large Software Engineering for Economists \\(7,610,1.00) \par}
\vspace{2.0cm}
{\itshape\large Alen Stepic - 11-475-258 \\Dimitrios Koumnakes - 10-613-370 \\Joël Sonderegger - 11-495-488 \\Severin Kranz - 13-606-355 \\Chi Xu - 16-300-915 \par}
	\begin{spacing}{1}
	\vspace{1.2cm}
	{Fall Term 2017 \par}
	\vspace{1.2cm}
	Supervisor:\\
	{Prof. Dr. Philipp Zahn\\ FGN HSG\\ Varnbüelstrasse 19\\ 9000 St. Gallen \par}
	\end{spacing}
\vfill
{\large \today\par}
\end{titlepage}
    
\clearpage
    
%************************** Abstract ************************** 
\begin{abstract}
\pagenumbering{Roman}
Based on previous research in the stock market, this paper uses a Twitter sentiment analysis to determine the dynamics between the amount of Twitter activity regarding Bitcoin and the market price of Bitcoin. The analysis is based on two different data sets. Twitter activity has been measured by the number of tweets containing the term "bitcoin" while Bitcoin prices were measured in the Bitcoin price index. The Vector Autoregression (VAR) model was used for the statistical analysis and the interpretation of the results.\\

The evaluation of the econometric model showed that changes in Twitter activity on Bitcoin have only a minimal and insignificant effect on the future price development of Bitcoin. This result must be understood with caution, because it only applies for the specific time interval of 8 days in which Twitter data have been gathered. Further on, researchers are being encouraged to elaborate on our observations, by applying our methodology over a longer time period and for additional cryptocurrencies. \\

\end{abstract}

\clearpage

%************************** Contenttable Page ************************** 
\tableofcontents

\clearpage

%************************** List of Figures ************************** 	
\listoffigures\bigskip\bigskip

\listoftables
 	   
 	   
\section*{List of Abbreviations} 
\begin{acronym}[ASECRETTT]
\acro{ADF}{Augmented Dickey-Fuller}
\acro{AIC}{Akaike Information Criterion}
\acro{EMH}{Efficient Market Hypothesis}
\acro{etc}{et cetera}
\acro{HQIC}{Hannan–Quinn information criterion}
\acro{SBIC}{Schwarz-Bayesian Information Criterion}
\acro{VAR}{Vector Autoregression}
\end{acronym}

\clearpage

%************************** Chapter 1 - Introduction **************************
\begin{spacing}{1.2}
\cleardoublepage\pagenumbering{arabic}
\section{Introduction}
\label{sec:intro}

\textcite[p.~388]{malkiel1970efficient} introduced the Efficient Market Hypothesis (EMH), where they claim that under certain market conditions actual prices include all information. This hypothesis is broadly accepted in the financial world. The digitalisation leads to an increased network effect, where information can be shared over the globe. Based on \citeauthor{mao2015quantifying} (\citeyear[][p.~3]{mao2015quantifying}, as cited in \cite[][pp.~175--195]{shiller2015irrational}; \cite[][pp.~279]{kahneman2013prospect}) the EMH fails to address the behavioral and emotional role of investors. The large fluctuation of the Bitcoin prices in the last year (mostly increasing) and the fact that Twitter has become a very popular social network, where people interact, led to different research in the field of sentimental analysis with the focus on Bitcoin and Twitter. \textcite[p.~18]{mao2015quantifying}, claim that sentiment analysis using Twitter tweets which contain the buzzword "bullishness" can indeed be used as a sentiment indicator for stock prices. As the modern economy becomes more digitised, cryptocurrencies, as a financial asset, draw more attention. This paper seeks to shed light on whether the tweets can be used to predict Bitcoin prices.

\subsection{Research Question and Goal of the Paper}
\label{sec:ResearchQandGoal}

As existing research focuses mostly on the sentimental analyses of tweet content, this paper aims to examine existence of a correlation between the price development of Bitcoin and the frequency of Bitcoin topics on Twitter. The goal should be reached by answering the following research question: \\
To what extent does the number of tweets about Bitcoin correlate with the price movement of Bitcoin?


\subsection{Methodology}
\label{sec:Methodology}
To answer the above mentioned research question, a scientific approach has been applied. The paper consists out of three parts. First, the relevance of the topic is pointed out by the introduction, the research question, methodology and the scope (see chapter \ref{sec:intro}). Secondly, the related theory has been analysed to provide the reader with the necessary background information by briefly introducing both Bitcoin and Twitter in context of price prediction and behavioral finance. The covered research is based on forward research in regard to the stock market and cryptocurrencies with a special focus on sentiment analysis and Bitcoin \ref{sec:Background}. Further on,  Vector Autoregression (VAR) approach is introduced and its constrains in context of the paper are being outlined (see chapter \ref{sec:EconometricModellingandResults}). The third part contains the discussion of the results. The computation is based on a data set, which was gathered by Python scripts for the time period December 20th - 27th. Since the results are based on the observation of a short period, they have to be interpreted carefully. The data set consists of two data sources with the following characteristics: (1) Twitter tweets which contain the buzzword "bitcoin" and (2) those containing historical Bitcoin prices. The two sources have been aggregated by using Python. The output of the aggregation was further processed with STATA. The third part of the paper is discussed in the chapter \ref{sec:EconometricModellingandResults} followed by a brief conclusion in chapter \ref{sec:Conclustion}.

\subsection{Scope}
\label{sec:Scope}
The paper's focus lies on the scientific discussion of empirical test results. The research question has consciously been narrowed, in order to build up on a existing methodology for sentiment analysis in the stock market while simultaneously extending the research scope in the area of behavioral finance and sentiment analysis for cryptocurrencies. The short time frame of our project further limited the period of data collection which has a direct impact on the reliability of our results.\\
Further on, this paper should act as a blueprint for further research in the area of Twitter sentiment analysis for cryptocurrencies. Therefore, the accompanying project documentation can be used as a guideline for further research.

\clearpage

%************************** Chapter 2 - Background and Data Collection **************************
\section{Background and Data Collection}
\label{sec:Background}
\subsection{Bitcoin}
In the aftermath of the global financial crisis, \textcite[p.~1]{Nakamoto2008Bitcoin} claimed, that a purely peer-to-peer version of electronic cash could bypass financial institutions as third parties for commercial transactions. Starting from this whitepaper, Bitcoin {-} the first electronic payment system that relies on the cryptographic concatenation of ongoing transactions \footnote{Blockchain} has been developed (\cite[p.~1]{Nakamoto2008Bitcoin}).\\

Today, Bitcoin is the most popular of thousands of cryptocurrencies worldwide and increased in value over 1700 percent within the last year according to Coinmarketcap\footnote{https://coinmarketcap.com/}. Even if the \textit{daily transaction volume}\footnotemark[2] increased rapidly over time, this does not necessarily mean that Bitcoin is used for payment purposes. Even more, several authors argue that the Bitcoin price and transaction volumes are mainly pushed by speculative investments rather than actual usage as a currency (\cite{corbet2017datestamping}; \cite{Forbes2017}; \cite{yermack2013bitcoin}). \textcite{kristoufek2015main} points out that there are other influencing factors such as usage in trade, money supply and price level, that influence the Bitcoin price in the long term. However, evidence shows that the level of Bitcoin prices are clearly driven by investors’ interest in the digital currency. \textcite{kristoufek2015main} further explains the correlation is most evident in the long run.  During explosive increases or rapid declines of the price higher investors’ interest further boosts the movement in the direction (\cite{kristoufek2015main}). These findings are in line with other researchers (see \cite[]{garcia2014digital}; \cite[]{kondor2014rich}).\\

In previous research, \textcite{madan2015automated} applied \textit{machine learning} algorithms to predict Bitcoin prices. By using the \textit{bayesian regression}, \textcite{shah2014bayesian} achieved a return on investment of 89 percent over an investment period of 50 days. However, research in this area is still limited and most of them do not take into consideration the individuals' sentiments (\cite[p.~1]{colianni2015algorithmic}).\\

One core assumption of behavioral finance is that investors’ interests influence their behavior and therefore stock prices (\cite[p.~3]{mao2015quantifying}). A widely used approach to measure investors’ interests is the sentiment analysis of Twitter data, which will be covered in the following section.\\

\subsection{Twitter}
With the current growth of social networks, the number of global social media users reached 2.46 billion and is expected to increase to 3.02 billion by 2021 according to \textcite{Statista2018a}. \textcite[p.~12]{Gabriel2017socialmedia} define social networks as a loose connection of people in an online or internet community, respectively in a computer-based communication network. This paper exploits the possibility to analyse shared “thoughts, views and opinions” (\cite[p.~6]{stenqvist2017predicting}) provided by such communities or communication networks.\\

Twitter was founded in 2006 and gained rapidly worldwide popularity and reached 330 million active users in the third quarter of 2017 (\cite[]{twitterinc2018}; \cite[]{Statista2018b}). As a micro-blogging platform, Twitter suits the purpose of analysing investors’ interests due to its special characteristics. In comparison to other social networks, Twitter limits its posts (tweets) to relatively short messages of 140 characters. Tweets can contain observations, thoughts, links to various content, websites, uploaded pictures or videos. Special conventions such as hashtags “\#” or mentions “@” are used to reference searchable content or user profiles (\cite{Schmidt2018socialmedia}).\\

In this way, Twitter users create millions of posts that contain interests, opinions and informations in both a private and professional context.  Due to the semi-structured form, the message length restriction and classifying nature, \textcite[p.~6]{stenqvist2017predicting} argue, that “Twitter has become a gold mine for opinionated data”, which is further supported in the wide adoption of researchers to analyse sentiments.\\

In relation to Bitcoin, researchers already showed that analysing investors sentiments can lead to reasonable correlations between Twitter sentiments and prices of cryptocurrencies \textcite[p.~7]{stenqvist2017predicting} , by either using a \textit{polarity classification} {-} classifying the language in tweets as either positive or negative (see \cite{colianni2015algorithmic}) {-} or a \textit{lexicon based approach} {-} attributes predefined words to specific sentiment classifications (see \cite{stenqvist2017predicting}). Hence, this paper choses a different focus area by limiting its scope to the number of daily tweets that contain the term “bitcoin” without analysing investors interests any further. Based on analysis of the influence of Twitter bullishness on the stock market by \textcite{mao2015quantifying}, this paper further elaborates on the concept by applying the approach to Bitcoin prices.\\

		
\subsection{Data Collection}
For our data, we collected two separate time series datasets. One showing the Twitter activity regarding Bitcoin, with the second one showing the historical Bitcoin price development. Both datasets were collected hourly for the period from the 19.12.2017 starting at 20:00:00 UTC and ending on the 28.12.2017 at 10:00:00 UTC. We chose the hourly basis to conduct statistical analysis with a limited amount of observations as demanded by the feasibility of such a complex project. For the Twitter data we aggregated by each hour all tweets mentioning "bitcoin". We used prices at the last second of every hour to represent the hourly Bitcoin price.\\
The following figure displays the Twitter activity with regard to Bitcoin for the observed time period. The first plot shows the hourly change of the aggregate number of tweets containing "bitcoin". In the second and third plots we respectively did a first difference and a logarithmic first difference transformation of our main variable, in case it is needed in the later analysis. The graphs show overall a cyclical movement around 1200 tweets per hour and a single outburst befalling on 22.12.2017 between 15:00:01 and 21:00:00, with highest value being 4622 tweets in one hour.\\

\begin{figure}[H]
	\begin{subfigure}{.3\textwidth}
	\centering
	\includegraphics[width=1.12\textwidth]{stata_export_graphs/graph_plot_nr_tweets.png}
	\caption{Number of Tweets regarding Bitcoin}
	\end{subfigure}\hfill
	\begin{subfigure}{.3\textwidth}
	\centering
	\includegraphics[width=1.12\textwidth]{stata_export_graphs/graph_plot_df_nr_tweets.png}
	\caption{First difference of number of Tweets}
	\end{subfigure}\hfill
	\begin{subfigure}{.3\textwidth}
	\centering
	\includegraphics[width=1.12\textwidth]{stata_export_graphs/graph_plot_log_df_nr_tweets.png}
	\caption{Log first difference of number of Tweets}
	\end{subfigure}
\caption{Twitter variables development}
\end{figure}

Next we display the time series data on Bitcoin price development for the observed time period. The first plot shows the hourly changes of Bitcoin prices. The second and third plots are once again first difference transformations of the main variable. In this graph we observe a notably volatile movement with no specific direction. This contradicts the general rising trend of Bitcoin prices of the past months. The problem may indeed arise if we happen to select a short period of one week when Bitcoin price movement behaves less bullishly as usual, in other words, the lack of articulated trend may be attributed to \textit{selection bias}.\\

\begin{figure}[H]
	\begin{subfigure}{.3\textwidth}
	\centering
	\includegraphics[width=1.12\textwidth]{stata_export_graphs/graph_plot_bpi.png}
	\caption{Hourly development of Bitcoin prices}
	\end{subfigure}\hfill
	\begin{subfigure}{.3\textwidth}
	\centering
	\includegraphics[width=1.12\textwidth]{stata_export_graphs/graph_plot_df_bpi.png}
	\caption{First difference of Bitcoin prices}
	\end{subfigure}\hfill
	\begin{subfigure}{.3\textwidth}
	\centering
	\includegraphics[width=1.12\textwidth]{stata_export_graphs/graph_plot_log_df_bpi.png}
	\caption{Log first difference of Bitcoin prices}
	\end{subfigure}
\caption{Bitcoin price development}
\end{figure}
	
\clearpage

%************************** Chapter 3 - Econometric Modelling and Results **************************
\section{Econometric Modelling and Results}
\label{sec:EconometricModellingandResults}
As stated in the beginning, our purpose is to examine the underlying relationship between the number of tweets containing "bitcoin" and Bitcoin price fluctuations. From our data collection, we are left with two time series datasets representing two variables. A broadly used statistical method to simultaneously analyze multiple time series is the Vector Autoregression (VAR) approach. In this method, the endogenous variables are determined both by their own historical values and by the historical values of the other endogenous variables (\cite[pp.~4--5]{luetkepohl2007new}).\\
To generate the following econometric results we used the statistical software STATA.

\subsection{Preparing for VAR: Stationarity Test}
Before estimating a VAR model, the time series data's stationarity must be confirmed. Namely, we need to demonstrate that the means and variances are constant over time, and that the dataset does not show any trending behavior. Non-stationary data can lead to an inaccurate estimation. To conduct stationarity, we use the Augmented Dickey-Fuller (ADF) test.\\

We shall test the stationarity for the "crude" Twitter data (variable name: nr\_of\_tweets), and since the variable may not pass the stationarity test, we also check for the stationarity of the first difference (df\_nr\_of\_tweets) and the logarithmic first difference (log\_df\_nr\_of\_tweets) transformations. The results of the augmented Dickey-Fuller (ADF) test are listed in the following table:\\

\begin{figure}[H]
\centering
\includegraphics[scale=0.85]{stata_export_graphs/ADF_twitter_variables.png}
\caption{Twitter variables stationarity test}
\label{fig:3}
\end{figure}
	
From the results we can see that only the first variable is non-stationary, since the absolute value of the Test Statistic (3.152) is smaller than the absolute value of the 1\% critical value (3.475), the null hypothesis of non-stationarity cannot be rejected as a consequence. The second variable is stationary (14.167 $>$ 3.475) and from here onwards we shall proceed the analysis with only the first difference tweet count.\\

Out of the similar concern stated above, we shall check for stationarity for the second variable, the Bitcoin prices (variable name: bpi\_closing\_price). As before, we also include the first difference Bitcoin prices (variable name: df\_bpi\_closing\_price) and the logarithmic first difference (variable name: log\_df\_bpi\_closing\_price) transformations. The ADF test for these variables gives the following results:\\

\begin{figure}[H]
\centering
\includegraphics[scale=0.85]{stata_export_graphs/ADF_bpi_variables.png}
\caption{Bitcoin variables stationarity test}
\label{fig:4}
\end{figure}
	
Once again, the test result suggests we use the first difference transformation of the variable Bitcoin prices, for it succeeds best the ADF test.

\subsection{Preparing for VAR: Lag-Specification}
To select the optimal number of lags to use for the VAR regression, we have to check for various information criteria. Information criteria measure the trade-off between model fit and parsimony, giving the optimal number of lags to use (\cite[p.~27]{brandtwilliams2007}). The calculations of these criteria for our specific data set are shown in the following table.\\

\begin{figure}[H]
\centering
\includegraphics[scale=0.85]{stata_export_graphs/LAG_crit_df_nr_tweets_df_bpi.png}
\caption{Lag Specification}
\label{fig:5}
\end{figure}

As can be seen, STATA calculates and presents various information criteria (AIC\footnote{Akaike Information Criterion}, HQIC\footnote{Hannan–Quinn information criterion}, SBIC\footnote{Schwarz-Bayesian Information Criterion}). The asterisk indicates the optimal lag length to use for the regression analysis. In this case all criteria suggest a lag length of 1.
		
\subsection{VAR Regression Model}
After determining the optimal lag length the next step is to build the vector autoregression model (VAR model).\\
For this analysis, we use a basic unrestricted VAR model which consists of two endogenous variables, T\textsubscript t which repesents the variable {\itshape df\_nr\_of\_tweets} in our data set and B\textsubscript t which represents {\itshape df\_bpi\_closing\_price}. As defined in the previous section the selected otpimal time lag is 1. Given these information, the model equations can be written as follows:

\begin{align}
T_t = c_1 + a_{11}T_{t-1} + a_{12}B_{t-1} + \epsilon_1 \\
B_t = c_2 + a_{21}T_{t-1} + a_{22}B_{t-1} + \epsilon_2
\end{align}

This model shows that the current value of our endogenous variables is determined by its own past values, the past values of the other variable and an error term. The second equation is directly relevant to our research question, as it allows to look into the correlation between the present Bitcoin price and the past tweet count.\\

After running the above VAR regression in STATA we get following results:
	
\begin{table}[H]
\centering
	\begin{tabular}{lcc} \hline
	 & (T\textsubscript{t}) & (B\textsubscript{t}) \\
	VARIABLES & df\_nr\_of\_tweets & df\_bpi\_closing\_price \\ \hline
	 &  &  \\
	L.df\_nr\_of\_tweets (T\textsubscript{t-1}) & 0.00999 (a\textsubscript{11}) & 0.00463 (a\textsubscript{21}) \\
	 & (0.0693) & (0.0104) \\
	L.df\_bpi\_closing\_price (B\textsubscript{t-1}) & -0.675 (a\textsubscript{12}) & -0.125* (a\textsubscript{22}) \\
	 & (0.469) & (0.0707) \\
	Constant  & -88.34 (c\textsubscript{1}) & -18.48 (c\textsubscript{2}) \\
	 & (177.9) & (26.79) \\
	 &  &  \\
	 Observations & 205 & 205 \\ \hline
	\multicolumn{3}{c}{ Standard errors in parentheses} \\
	\multicolumn{3}{c}{ *** p$<$0.01, ** p$<$0.05, * p$<$0.1} \\
	\end{tabular}
\caption{Var Regression}
\end{table}

The equations of our VAR model are displayed vertically in this table. As said before the relevant equation in the model is the second one which is listed on the right column (B\textsubscript{t}). The coefficient of 0.00463 (a\textsubscript{21}) implies that a change of tweets by 1 tweet, changes the Bitcoin price by 0.005 \$. This effect is very small and statistically insignificant (p value $>$ 0.1). This answers our research question with the following statement:\\ The amount of tweets mentioning Bitcoin has no significant effect on the price development of Bitcoin.
		
\subsection{Granger Causality}
To assess the reliability (causality) of the previous results we use the Granger causality test. This test confirms if the statements by an estimated regression are valid. The following table shows the results of the Granger causality test for our VAR regression:\\

\begin{figure}[H]
\centering
\includegraphics[scale=0.85]{stata_export_graphs/granger_test.png}
\caption{Granger causality test}
\label{fig:6}
\end{figure}

The relevant results for our regression are displayed in the lower row. The Null hypothesis here is that the lagged (lag 1) values of the number of tweets does not cause a change in the Bitcoin prices. The Alternative hypothesis would then be that the lagged values of the number of tweets does cause a change in Bitcoin prices. As the results show, the Probability value is 65.7\% which is clearly more than 5\% and constrain us to accept the Null hypothesis and conclude that the regression results stated above are valid.\\

\clearpage

%************************** Chapter 4 - Conclusion **************************
\section{Conclusion}
\label{sec:Conclustion}
Understanding, describing and predicting the current phenomenon of cryptocurrencies has been the objective of many researchers. A variety of methods and approaches have been applied and mixed results have been produced. Our goal for this project was to investigate the possibility of the existence of a correlation between Twitter activity regarding Bitcoin and the price evolution of Bitcoin. After collecting hourly data for 8 days, we generated two time-series datasets and decided to perform a statistical analysis by using the Vector Autoregression (VAR) approach.\\

According to our regression results, changes in Twitter activity caused by a higher or lower number of tweets about Bitcoin, have a very small and insignificant effect on future changes of Bitcoin price. The validity of this statement was also verified and confirmed by the Granger causality test. At this point, however, it must be pointed out that only a small period of time has been considered in this analysis and the results cannot be generalized for the overall trend of cryptocurrencies in the recent time or for the future. We are also aware, that we may have observed a period of abnormal behavior of Bitcoin prices compared to the trend of the last months.\\

We encourage further research in the same direction as this paper could serve as a blueprint together with the accompanying process documentation and would be interested to see an analysis with observations over a longer period. 


\clearpage
		
\end{spacing}

\clearpage

%************************** Bibliography **************************
\section{References}
\printbibliography[heading=none]

\clearpage

%************************** Declaration of Authorship **************************
\section{Declaration of Authorship}
We hereby declare,
\begin{itemize}
\item that we have written this thesis without any help from others and without the use of documents and aids other than those stated above;
\item that we have mentioned all the sources used and that we have cited them correctly according to established academic citation rules;
\item that we have acquired any immaterial rights to materials we may have used such as images or graphs, or that we have produced such materials ourself;
\item that the topic or parts of it are not already the object of any work or examination of another course unless this has been explicitly agreed on with the faculty member in advance and is referred to in the thesis;
\item that we are aware that our work can be electronically checked for plagiarism and that we hereby grant the University of St.Gallen copyright in accordance with the Examination Regulations in so far as this is required for administrative action;
\item that we are aware that the University will prosecute any infringement of this declaration of authorship and, in particular, the employment of a ghostwriter, and that any such infringement may result in disciplinary and criminal consequences which may result in our expulsion from the University or us being stripped of our degree.
\end{itemize}

\begin{flushleft}
Alen Stepic - 11-475-258\\\bigskip\bigskip  
Dimitrios Koumnakes - 10-613-370\\\bigskip\bigskip
Severin Kranz - 13-606-355\\\bigskip\bigskip
Joël Sonderegger - 11-495-488\\\bigskip\bigskip
Chi Xu - 16-300-915

By submitting this academic term paper, we confirm through my conclusive action that we are submitting the Declaration of Authorship, that we have read and understood it, and that it is true.
\end{flushleft}

\clearpage

\end{document}
%************************** DOCUMENT_ENDS_HERE **************************
